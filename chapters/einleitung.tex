\section{Einleitung}		   	% 
Dies ist normaler Text....


\label{referenzKey} 					   	% Einen Referenzpunkt setzen
%											% 
%% Text Styles								%
\gqq{Anführungszeichen} 				\\ 	% 
\textbf{Fettgedruckt}					\\ 	% 
\textit{Kursiv}							\\ 	% 
\underline{Unterstrichen}				\\ 	% 
\\											% Zeilenumbruch
%											%
%% Text Elements							%
\myboxquote{Dieser Text steht in einer Box} % Für längere Zitate geeignet
\begin{itemize}								% Beginn einer Aufzählung
	\item erster Punkt						% Aufzählungspunkt
	\item Zweiter Punkt						% Aufzählungspunkt
\end{itemize}								% Ende der Aufzählung

%											% 
%% Glossar									%
\gls{NEO-FFI} 								\\ 	% Glossar eintrag
\glspl{NEO-FFI} 							\\ 	% Plural des Glossar eintrags
\gls{NEO-FFI} 								\\ 	% Wenn Glossar eintrag erstamlig verwendet wird
\glspl{NEO-FFI} 							\\ 	% Wird die Bezeichnung ausgeschrieben, 
%											% anschließend wir abgekürzt
%											% 
%% Quellen Referenzierung 					%
\textcite{BiBkey}
%											% 
%% Label Referenzierung 					%
(siehe \mypageref{referenzKey})			\\ 	Referenz auf \label key mit Seitenangabe
Abb. \ref{referenzKey}					\\	 Referenz auf \label ohne Seitenangabe
%											%
%% Weitere Hinweise
Mehrmals Kompilieren hilft manchmal bei Problemen.\\
Alles außer den Ordnern und der \{dateiname\}.tex aus dem Root-Verzeichnis löschen und mehrmals neu kompilieren löst hartnäckigere Fehler.