%%
%%                   LaTeX Vorlage für die Studenten der
%%              Fachhochschule für angewandtes Management
%%
%%  Die Vorlage orientiert sich an den Gestaltungsrichtlinien der Fachhochschule für angewandtes Management
%%
%%
%% Ersteller:        Julian Beisenkötter
%% letzten Änderung: 20. Januar 2016
%% Version:			 1.0.0
%%
%%
%% Wichtiger Hinweis zur Verwendung dieser Vorlage:
%%
%% Konfiguration für TeXstudio:
%% Preferences/erzeugen/Benutzerbefehle:
%% txs:///bibtex | txs:///makeindex "%.idx" -t "%.ilg" -o "%.ind" | txs:///makeindex -g -s "styles/stichwortverzeichnis.ist" % | txs:///makeindex "%.nlo" -s nomencl.ist -o "%.nls" | txs:///makeindex  -s "%.ist" -t "%.glg" -o "%.gls" "%.glo"
%% mit Tools/benutzer/ keykompilieren.
%%
%% Wichtiger Hinweis:
%%
%% Keine Änderungen an den Dateien im Verzeichnis "pages" vornehmen. Für
%% die Arbeit beziehen sich alle Änderungen auf diese Datei und die
%% Dateien im Verzeichnis "chapter".
%% Für die Erstellung des Literaturverzeichnises empfiehlt sich die Ver-
%% wendung von JabRef (http://jabref.sourceforge.net).
%% oder Mendeley (https://www.mendeley.com/).
%% Die Datei ist unter dem Namen literatur.bib im Verzeichnis "literatur" zu speichern.
%%

\documentclass[a4paper,12pt]{article}                                         									   % Schriftgröße, Layout, Papierformat, Art des Dokumentes
\usepackage[left=2.5cm,right=2.5cm,top=3cm,bottom=1cm,includehead]{geometry}      % Einstellungen der Seitenränder
\usepackage{ngerman}                                                          											% neue Rechtschreibung
\usepackage[ngerman]{babel}                                                   										 % deutsche Silbentrennung
\usepackage[utf8]{inputenc}                                                   											% Umlaute
\usepackage[hyperfootnotes=false, pdfborder={0 0 0}]{hyperref}                                   	% pfd-Output [Fußnoten nicht verlinken]
\usepackage[nottoc]{tocbibind}                                                										  % Inhaltsverzeichniserweiterung (Inhaltsverzeichnis selbst ausblenden)
\usepackage{makeidx}                                                          											 % Index
\usepackage[intoc]{nomencl}                                                   										  % Abkürzungsverzeichnis
\usepackage{fancyhdr}           	                                          											 % Fancy Header

\usepackage{amsmath}                                                          											% Zurücksetzen der Tabellen- und Abbildungsnummerierung je Sektion
\usepackage[labelfont=bf,aboveskip=1mm]{caption}                              							% Bild- und Tabellenunterschrift (fett)
\usepackage[onehalfspacing]{setspace}                                                         					% Zeilenabstand (vor footmisc laden!)
\usepackage[bottom,multiple,hang,marginal]{footmisc}                          							% Fußnoten [Ausrichtung unten, Trennung durch Seperator bei mehreren Fußnoten]
\usepackage{wrapfig}																									   % Umfließende Grafiken und Tabellen
\usepackage{graphicx}                                                         											  % Grafiken
\usepackage{tabularx}                                                     											       % erweiterte Tabellen
\usepackage{longtable}                                                 												      % mehrseitige Tabellen
\usepackage{color}                                                            												% Farben
\usepackage{enumitem}                                                         											% Befehl setlist (Zeilenabstand für itemize Umgebung auf 1 setzen)
\usepackage{listings}                                                         												% Quelltexte
\usepackage{zref}                                                             												 % Verweise (Anhangsverweise)
\usepackage[toc,style=altlist,translate=false]{glossaries}                    								% Glossar (nach hyperref, inputenc, babel und ngerman)
\usepackage{glossaries-babel}     											  										 % Glossar: Übersetzung im TOC
\usepackage{wasysym}
\usepackage{amssymb}
% ORPHANS
\usepackage{xspace}
\widowpenalty=300
\clubpenalty=300
\usepackage{multicol}
\usepackage{apacite}

%%%%%%%%%%%%%%%%%%%%%%%%%%%%%%%%%%%%%%%%%%%%%%%%%%%%%%%%%%%%%%%%%%%%%%%%%%%%%
%%                                                                         %%
%% \/   \/      Bitte hier die Änderungen zur Arbeit vornehmen     \/   \/ %%
%%                                                                         %%
%%%%%%%%%%%%%%%%%%%%%%%%%%%%%%%%%%%%%%%%%%%%%%%%%%%%%%%%%%%%%%%%%%%%%%%%%%%%%

%%%%%%%%%%%%%%%%%%%%%%% Definitionen bzgl. der Arbeit %%%%%%%%%%%%%%%%%%%%%%%
\def\myType{0}           % [0=Facharbeit | 1=Abschlussarbeit]

%%%%%%%%%%%%%%%%%%%% Folgende Angaben für: Facharbeit %%%%%%%%%%%%%%%%%%%%
\def\myModul{Empirische Forschung in Wissenschaft und Praxis}

%%%%%%%%%%%%%%%%%%%% Folgende Angaben für: Abschlussarbeit %%%%%%%%%%%%%%%%%%%%
\def\myProjectType{Masterarbeit} % [Bachelorarbeit | Masterbeit]

%%%%%%%%%%%%%%%%%%%% Folgende Angaben für: Alle Arbeiten %%%%%%%%%%%%%%%%%%%%
\def\myTopic{Auswirkungen von ehrenamtlichem Engagement auf das Persönlichkeitsinventar von jungen Erwachsenen}
\def\mySubTopic{Untersuchung mittels NEO-Fünf-Faktoren-Inventar (NEO-FFI)}
\def\myEndDate{21.03.2016}


\def\myAutor{Romana Amelung}
\def\myStreet{Seeburger Weg 14.}
\def\myCity{13581 Berlin}
\def\myMarikelnummer{46891}
\def\myTel{+49 157 73559462}

\def\mySemester{1}
\def\mySemDate{Wintersemester 2015}
\def \myCampus{Berlin}

\def\myDozTitle{Professorin}
\def\myDoz{Prof. Dr. Petra Schepler}


%%%%%%%%%%%%%%%%%%%%%%%%%%%%%%%%%%%%%%%%%%%%%%%%%%%%%%%%%%%%%%%%%%%%%%%%%%%%%
%%                                                                         %%
%% /\   /\         Ab hier keine Änderungen mehr vornehmen         /\   /\ %%
%%                                                                         %%
%%%%%%%%%%%%%%%%%%%%%%%%%%%%%%%%%%%%%%%%%%%%%%%%%%%%%%%%%%%%%%%%%%%%%%%%%%%%%

%%%%%%%%%%%%%%%%%%%%%%%% Eigene Farbwerte definieren %%%%%%%%%%%%%%%%%%%%%%%%
\definecolor{boxgray}{gray}{0.9}         % Hintergrundfarbe für Zitatboxen
\definecolor{commentgray}{gray}{0.5}     % Grau für Kommentare in Quelltexten
\definecolor{darkgreen}{rgb}{0,.5,0}     % Grün für Strings in Quelltexten

%%%%%%%%%%%%%%%%%%%%%%%% Eigene Kommandos definieren %%%%%%%%%%%%%%%%%%%%%%%%
% Definition von \gqq{#1: text}: Text in Anführungszeichen
\newcommand{\gqq}[1]{\glqq #1\grqq}

% Definition von \footref{#1: label}
% Verweis auf bereits existierende Fußnoten mittels
\providecommand*{\footref}[1]{
	\begingroup
		\unrestored@protected@xdef\@thefnmark{\ref{#1}}
	\endgroup
\@footnotemark}

% Definition von \mypageref{#1: label}
% Kombination aus \ref{#1: label} und \pageref{#1: label}
\newcommand{\mypageref}[1]{\ref{#1} \nameref{#1} auf Seite \pageref{#1}}

% Definition von \myboxquote{#1: text}
% grau hinterlegte Quotation-Umgebung (für Zitate)
\newcommand{\myboxquote}[1]{
	\begin{quotation}
		\colorbox{boxgray}{\parbox{0.78\textwidth}{#1}}
	\end{quotation}
	\vspace*{1mm}
}

\makeatletter
\zref@newprop*{appsec}{}
\zref@addprop{main}{appsec}

% Definition von \applabel{#1: label}{#2: text}
% von \appsec{#1: text}{#2: label} zur Erzeugung des Labels verwendet)
\def\applabel#1#2{%
	\zref@setcurrent{appsec}{#2}%
	\zref@wrapper@immediate{\zref@label{#1}}%
}

% Definition von \appref{#1: label}
% anstelle \ref{#1: label} zum referenzieren von Anhängen verwenden)
\def\appref#1{%
	\hyperref[#1]{\zref@extract{#1}{appsec}}%
}
\makeatother

% Definition von \appsection{#1: text}{#2: label}
% Ersetzt \section{#1: text} und \label{#2: label} für Anhänge)
\newcommand{\theappsection}[1]{Anhang \Alph{section}:~\protect #1}
\newcommand{\appsection}[2]{
	\addtocounter{section}{1}
	\phantomsection
	\addcontentsline{toc}{section}{\theappsection{#1}}
	\markboth{\theappsection{#1}}{}

	\section*{\theappsection{#1}}
	\applabel{#2}{Anhang \Alph{section}}
	\label{#2}
}

%%%%%%%%%%%%% Index, Abkürzungsverzeichnis und Glossar erstellen %%%%%%%%%%%%
\makeindex
\makenomenclature
\makeglossaries

% Festlegung der Art der Zitierung (Havardmethode: Abkuerzung Autor + Jahr) %
\bibliographystyle{apacite}

%%%%%%%%%%%%%%%%%%%%%%%%%%%%%%% PDF-Optionen %%%%%%%%%%%%%%%%%%%%%%%%%%%%%%%%
\hypersetup{
	bookmarksopen=false,
	bookmarksnumbered=true,
	bookmarksopenlevel=0,
	pdftitle=\myTopic,
	pdfsubject=\myTopic,
	pdfauthor=\myAutor,
	pdfborder=0,
	pdfstartview=Fit,
	pdfpagelayout=SinglePage
}

%%%%%%%%%%%%%%%%%%%%%%%%%%%% Kopf- und Fußzeile %%%%%%%%%%%%%%%%%%%%%%%%%%%%%
\pagestyle{fancy}
\fancyhf{}
\fancyhead[R]{\thepage}                         % Kopfzeile rechts bzw. außen
\renewcommand{\headrulewidth}{0.5pt}            % Kopfzeile rechts bzw. außen

%%%%%%%%%%%%%%%%%%%%%%%%% Layout und Beschriftungen %%%%%%%%%%%%%%%%%%%%%%%%%
\onehalfspacing                % Zeilenabstand: 1.5 (Standard: 1.2)
\setlist{noitemsep}            % Zeilenabstand für items auf 1 setzen

\addto\captionsngerman{        % Tabllen- und ildungsunterschriften ändern
  \renewcommand{\figurename}{Abb.}
  \renewcommand{\tablename}{Tab.}
}
\numberwithin{table}{section}                               % Tabellennummerierung je Sektion zurücksetzen
\numberwithin{figure}{section}                              % Abbildungsnummerierung je Sektion zurücksetzen
\renewcommand{\thetable}{\arabic{section}.\arabic{table}}   % Tabellennummerierung mit Section
\renewcommand{\thefigure}{\arabic{section}.\arabic{figure}} % Abbildungsnummerierung mit Section
\renewcommand{\thefootnote}{\arabic{footnote}}              % Sektionsbezeichnung von Fußnoten entfernen

\renewcommand{\multfootsep}{, }                             % Mehrere Fußnoten durch ", " trennen

%%%%%%%%%%%%%%%%%%%%%%%%%%%%%%% Listingstyle %%%%%%%%%%%%%%%%%%%%%%%%%%%%%%%%
\lstset{
	basicstyle=\ttfamily\scriptsize,
	commentstyle=\color{commentgray}\textit,
	showstringspaces=false,
	stringstyle=\color{darkgreen},
	keywordstyle=\color{blue},
	numbers=left,
	numberstyle=\tiny,
	stepnumber=1,
	numbersep=15pt,
	tabsize=2,
	fontadjust=true,
	frame=single,
	backgroundcolor=\color{boxgray},
	captionpos=b,
	linewidth=0.94\linewidth,
	xleftmargin=0.1\linewidth,
	breaklines=true,
	aboveskip=16pt
}

%%%%%%%%%%%%%%%%%%%%%%%%%%%%%%%%%%%%%%%%%%%%%%%%%%%%%%%%%%%%%%%%%%%%%%%%%%%%%
%%                                                                         %%
%% \/   \/      Bitte hier nur bei Bedarf Änderungen vornehmen     \/   \/ %%
%%                                                                         %%
%%%%%%%%%%%%%%%%%%%%%%%%%%%%%%%%%%%%%%%%%%%%%%%%%%%%%%%%%%%%%%%%%%%%%%%%%%%%%

%Seiten und Kapitel einbinden
\begin{document}
	\pagenumbering{Roman}
	% Das Titelblatt wird automatisch ausgewählt. Keine Änderung hier
	\ifcase\myType
		\begin{titlepage}
	\noindent
	Fachhochschule für angewandtes Management in \myCampus\\
	Fachbereich Wirtschaftspsychologie\\
	\mySemDate \\%Wintersemester 2015
	Modul: \myModul \\%Leadership
	\myDozTitle: \myDoz \\%Angelika Adam-Fendel
	\begin{center}
		\vspace*{2cm}
		\LARGE\bf\myTopic\\
		\Large\rm\mySubTopic\\
		\vspace*{2cm}
		Vorgelegt von \myAutor\\
		\vspace*{3cm}
		%\vfill
	\end{center}
	
	\noindent
	\begin{tabularx}{\linewidth}{lXl}
		\myStreet             				&& \mySemester. Semester 								\\
		\myCity 	  						 && Martrikelnummer: \myMarikelnummer      		\\
		Tel: \myTel           				 &&																		\\
	\end{tabularx} 
	\vspace*{2cm}
	\begin{center}
		Tag der Einreichung:\\ \myEndDate\\
	\end{center}
\end{titlepage}
\newpage
\setcounter{page}{2}
	\or
		\begin{titlepage}
	\noindent
	Fachhochschule für angewandtes Management in \myCampus\\
	Fachbereich Wirtschaftspsychologie\\
	\mySemDate \\%Wintersemester 2015
	\begin{center}
		\vspace*{2cm}
		\Large\rm\myProjectType\\
		\vspace*{0.5cm}
		\LARGE\bf\myTopic\\
		\Large\rm\mySubTopic\\
		\vspace*{2cm}
		\large
		Vorgelegt von \\ \myAutor\\
		\mySemester. Semester\\
		Martrikel-Nr.: \myMarikelnummer
		\vspace*{4cm}
		%\vfill
	\end{center}
	
	\noindent
	\begin{tabularx}{\linewidth}{lXl}
		\myStreet             				&& Erstbetreuer: \myDoz 								\\
		\myCity 	  						 && Tag der Einreichung: \myEndDate               \\  		
		Tel: \myTel           				 &&																	  \\
	\end{tabularx} 

	\vfill
\end{titlepage}
\newpage
\setcounter{page}{2}
	\else
	\fi

	\pagestyle{fancy}
	\tableofcontents
\newpage

	\renewcommand{\nomname}{Abkürzungsverzeichnis}
\setlength{\nomlabelwidth}{.25\hsize}
\renewcommand{\nomlabel}[1]{#1 \dotfill}
\setlength{\nomitemsep}{-\parsep}

% Akronüme - Abkürzungen
\nomenclature{DHBW}{Duale Hochschule Baden-Württemberg}
%			Key  Abkürzung 	Ausgeschriebene Bezeichnung

\printnomenclature
\newpage

	\listoffigures
\newpage

	\listoftables
\newpage

	%Listingnummering je Sektion zurücksetzen
\numberwithin{lstlisting}{section}
%Listingnummerierung mit Section
\renewcommand{\thelstlisting}{\arabic{section}.\arabic{lstlisting}}
%Listingsverzeichnis in das Inhaltsverzeichnis aufnehmen.
\renewcommand{\lstlistingname}{Listingsverzeichnis}
\renewcommand{\lstlistoflistings}{\begingroup
\tocchapter
\tocfile{\lstlistingname}{lol}
\endgroup}

\lstlistoflistings
\newpage

\renewcommand{\lstlistingname}{Listing}

	\section*{Abstract}	
Dies ist dein Abstract!

	% Kapitel
	\pagestyle{fancy}
	\fancyhead[L]{\nouppercase{\leftmark}}                               % Kopfzeile links bzw. innen
	\pagenumbering{arabic}

%%%%%%%%%%%%%%%%%%%%%%%%%%%%%%%%%%%%%%%%%%%%%%%%%%%%%%%%%%%%%%%%%%%%%%%%%%%%%
%%                                                                         %%
%% \/   \/      Bitte hier die Änderungen zur Arbeit vornehmen     \/   \/ %%
%%                                                                         %%
%%%%%%%%%%%%%%%%%%%%%%%%%%%%%%%%%%%%%%%%%%%%%%%%%%%%%%%%%%%%%%%%%%%%%%%%%%%%%

	% Glossaryentries

% Glossareintrag
\newglossaryentry{NEO-FFI} % Key (Schlüsselwort zum Aufruf)
{
	name={NEO-FFI}, % Bezeichnung
	plural={NEO-FFIs}, % Pluralisierte Bezeichnung
	description={Neurotizismus Extraversion Offenheit für Erfahrungen Fünf Faktoren Indikator} % Erläuterung
}

% Akronüme - Abkürzungen
\newacronym{NEO-FFI}{NEO-FFI}{	} 
%			Key  Abkürzung 	Ausgeschriebene Bezeichnung

	% Glossar falls  benötigt


	\section{Einleitung}		   	% 
Dies ist normaler Text....


\label{referenzKey} 					   	% Einen Referenzpunkt setzen
%											% 
%% Text Styles								%
\gqq{Anführungszeichen} 				\\ 	% 
\textbf{Fettgedruckt}					\\ 	% 
\textit{Kursiv}							\\ 	% 
\underline{Unterstrichen}				\\ 	% 
\\											% Zeilenumbruch
%											%
%% Text Elements							%
\myboxquote{Dieser Text steht in einer Box} % Für längere Zitate geeignet
\begin{itemize}								% Beginn einer Aufzählung
	\item erster Punkt						% Aufzählungspunkt
	\item Zweiter Punkt						% Aufzählungspunkt
\end{itemize}								% Ende der Aufzählung

%											% 
%% Glossar									%
\gls{BPM} 								\\ 	% Glossar eintrag
\glspl{BPM} 							\\ 	% Plural des Glossar eintrags
\gls{ASM} 								\\ 	% Wenn Glossar eintrag erstamlig verwendet wird
\glspl{ASM} 							\\ 	% Wird die Bezeichnung ausgeschrieben, 
%											% anschließend wir abgekürzt
%											% 
%% Quellen Referenzierung 					%
Quellen verweise nach den Richtlinien der APA\\
\citeA{BiBkey}. inline und am ende. \\
\cite{BiBkey}
%											% 
%% Label Referenzierung 					%
(siehe \mypageref{referenzKey})			\\ 	Referenz auf \label key mit Seitenangabe
Abb. \ref{Ehrenamt}					\\	 Referenz auf \label ohne Seitenangabe
%											%
%% Weitere Hinweise
Mehrmals Kompilieren hilft manchmal bei Problemen.\\
Alles außer den Ordnern und der \{dateiname\}.tex aus dem Root-Verzeichnis löschen und mehrmals neu kompilieren löst hartnäckigere Fehler.
	\section{Kapitel}		   	
Dies ist ein Kapitel mit Beispieltext:\\
Lorem ipsum dolor sit amet, consetetur sadipscing elitr, sed diam nonumy eirmod tempor invidunt ut labore et dolore magna aliquyam erat, sed diam voluptua. At vero eos et accusam et justo duo dolores et ea rebum. Stet clita kasd gubergren, no sea takimata sanctus est Lorem ipsum dolor sit amet. Lorem ipsum dolor sit amet, consetetur sadipscing elitr, sed diam nonumy eirmod tempor invidunt ut labore et dolore magna aliquyam erat, sed diam voluptua. At vero eos et accusam et justo duo dolores et ea rebum. Stet clita kasd gubergren, no sea takimata sanctus est Lorem ipsum dolor sit amet. Lorem ipsum dolor sit amet, consetetur sadipscing elitr, sed diam nonumy eirmod tempor invidunt ut labore et dolore magna aliquyam erat, sed diam voluptua. At vero eos et accusam et justo duo dolores et ea rebum. Stet clita kasd gubergren, no sea takimata sanctus est Lorem ipsum dolor sit amet. 

Duis autem vel eum iriure dolor in hendrerit in vulputate velit esse molestie consequat, vel illum dolore eu feugiat nulla facilisis at vero eros et accumsan et iusto odio dignissim qui blandit praesent luptatum zzril delenit augue duis dolore te feugait nulla facilisi. Lorem ipsum dolor sit amet, consectetuer adipiscing elit, sed diam nonummy nibh euismod tincidunt ut laoreet dolore magna aliquam erat volutpat. 

Ut wisi enim ad minim veniam, quis nostrud exerci tation ullamcorper suscipit lobortis nisl ut aliquip ex ea commodo consequat. Duis autem vel eum iriure dolor in hendrerit in vulputate velit esse molestie consequat, vel illum dolore eu feugiat nulla facilisis at vero eros et accumsan et iusto odio dignissim qui blandit praesent luptatum zzril delenit augue duis dolore te feugait nulla facilisi. 

Nam liber tempor cum soluta nobis eleifend option congue nihil imperdiet doming id quod mazim placerat facer possim assum. Lorem ipsum dolor sit amet, consectetuer adipiscing elit, sed diam nonummy nibh euismod tincidunt ut laoreet dolore magna aliquam erat volutpat. Ut wisi enim ad minim veniam, quis nostrud exerci tation ullamcorper suscipit lobortis nisl ut aliquip ex ea commodo consequat. 

Duis autem vel eum iriure dolor in hendrerit in vulputate velit esse molestie consequat, vel illum dolore eu feugiat nulla facilisis. 

At vero eos et accusam et justo duo dolores et ea rebum. Stet clita kasd gubergren, no sea takimata sanctus est Lorem ipsum dolor sit amet. Lorem ipsum dolor sit amet, consetetur sadipscing elitr, sed diam nonumy eirmod tempor invidunt ut labore et dolore magna aliquyam erat, sed diam voluptua. At vero eos et accusam et justo duo dolores et ea rebum. Stet clita kasd gubergren, no sea takimata sanctus est Lorem ipsum dolor sit amet. Lorem ipsum dolor sit amet, consetetur sadipscing elitr, At accusam aliquyam diam diam dolore dolores duo eirmod eos erat, et nonumy sed tempor et et invidunt justo labore Stet clita ea et gubergren, kasd magna no rebum. sanctus sea sed takimata ut vero voluptua. est Lorem ipsum dolor sit amet. Lorem ipsum dolor sit amet, consetetur sadipscing elitr, sed diam nonumy eirmod tempor invidunt ut labore et dolore magna aliquyam erat. 

Consetetur sadipscing elitr, sed diam nonumy eirmod tempor invidunt ut labore et dolore magna aliquyam erat, sed diam voluptua. At vero eos et accusam et justo duo dolores et ea rebum. Stet clita kasd gubergren, no sea takimata sanctus est Lorem ipsum dolor sit amet. Lorem ipsum dolor sit amet, consetetur sadipscing elitr, sed diam nonumy eirmod tempor invidunt ut labore et dolore magna aliquyam erat, sed diam voluptua. At vero eos et accusam et justo duo dolores et ea rebum. Stet clita kasd gubergren, no sea takimata sanctus est Lorem ipsum dolor sit amet. Lorem ipsum dolor sit amet, consetetur sadipscing elitr, sed diam nonumy eirmod tempor invidunt ut labore et dolore magna aliquyam erat, sed diam voluptua. At vero eos et accusam et justo duo dolores et ea rebum. Stet clita kasd gubergren, no sea takimata sanctus. 

Lorem ipsum dolor sit amet, consetetur sadipscing elitr, sed diam nonumy eirmod tempor invidunt ut labore et dolore magna aliquyam erat, sed diam voluptua. At vero eos et accusam et justo duo dolores et ea rebum. Stet clita kasd gubergren, no sea takimata sanctus est Lorem ipsum dolor sit amet. Lorem ipsum dolor sit amet, consetetur sadipscing elitr, sed diam nonumy eirmod tempor invidunt ut labore et dolore magna aliquyam erat, sed diam voluptua. At vero eos et accusam et justo duo dolores et ea rebum. Stet clita kasd gubergren, no sea takimata sanctus est Lorem ipsum dolor sit amet. Lorem ipsum dolor sit amet, consetetur sadipscing elitr, sed diam nonumy eirmod tempor invidunt ut labore et dolore magna aliquyam erat, sed diam voluptua. At vero eos et accusam et justo duo dolores et ea rebum. Stet clita kasd gubergren, no sea takimata sanctus est Lorem ipsum dolor sit amet. 

Duis autem vel eum iriure dolor in hendrerit in vulputate velit esse molestie consequat, vel illum dolore eu feugiat nulla facilisis at vero eros et accumsan et iusto odio dignissim qui blandit praesent luptatum zzril delenit augue duis dolore te feugait nulla facilisi. Lorem ipsum dolor sit amet, consectetuer adipiscing elit, sed diam nonummy nibh euismod tincidunt ut laoreet dolore magna aliquam erat volutpat. 

Ut wisi enim ad minim veniam, quis nostrud exerci tation ullamcorper suscipit lobortis nisl ut aliquip ex ea commodo consequat. Duis autem vel eum iriure dolor in hendrerit in vulputate velit esse molestie consequat, vel illum dolore eu feugiat nulla facilisis at vero eros et accumsan et iusto odio dignissim qui blandit praesent luptatum zzril delenit augue duis dolore te feugait nulla facilisi. 

Nam liber tempor cum soluta nobis eleifend option congue nihil imperdiet doming id quod mazim placerat facer possim assum. Lorem ipsum dolor sit amet, consectetuer adipiscing elit, sed diam nonummy nibh euismod tincidunt ut laoreet dolore magna aliquam erat volutpat. Ut wisi enim ad minim veniam, quis nostrud exerci tation ullamcorper suscipit lobortis nisl ut aliquip ex ea commodo consequat. 

Duis autem vel eum iriure dolor in hendrerit in vulputate velit esse molestie consequat, vel illum dolore eu feugiat nulla facilisis. 

At vero eos et accusam et justo duo dolores et ea rebum. Stet clita kasd gubergren, no sea takimata sanctus est Lorem ipsum dolor sit amet. Lorem ipsum dolor sit amet, consetetur sadipscing elitr, sed diam nonumy eirmod tempor invidunt ut labore et dolore magna aliquyam erat, sed diam voluptua. At vero eos et accusam et justo duo dolores et ea rebum. Stet clita kasd gubergren, no sea takimata sanctus est Lorem ipsum dolor sit amet. Lorem ipsum dolor sit amet, consetetur sadipscing elitr, At accusam aliquyam diam diam dolore dolores duo eirmod eos erat, et nonumy sed tempor et et invidunt justo labore Stet clita ea et gubergren, kasd magna no rebum. sanctus sea sed takimata ut vero voluptua. est Lorem ipsum dolor sit amet. Lorem ipsum dolor sit amet, consetetur sadipscing elitr, sed diam nonumy eirmod tempor invidunt ut labore et dolore magna aliquyam erat. 

Consetetur sadipscing elitr, sed diam nonumy eirmod tempor invidunt ut labore et dolore magna aliquyam erat, sed diam voluptua. At vero eos et accusam et justo duo dolores et ea rebum. Stet clita kasd gubergren, no sea takimata sanctus est Lorem ipsum dolor sit amet. Lorem ipsum dolor sit amet, consetetur sadipscing elitr, sed diam nonumy eirmod tempor invidunt ut labore et dolore magna aliquyam erat, sed diam voluptua. At vero eos et accusam et justo duo dolores et ea rebum. Stet clita kasd gubergren, no sea takimata sanctus est Lorem ipsum dolor sit amet. Lorem ipsum dolor sit amet, consetetur sadipscing elitr, sed diam nonumy eirmod tempor invidunt ut labore et dolore magna aliquyam erat, sed diam voluptua. At vero eos et accusam et justo duo dolores et ea rebum. Stet clita kasd gubergren, no sea takimata sanctus. 

Lorem ipsum dolor sit amet, consetetur sadipscing elitr, sed diam nonumy eirmod tempor invidunt ut labore et dolore magna aliquyam erat, sed diam voluptua. At vero eos et accusam et justo duo dolores et ea rebum. Stet clita kasd gubergren, no sea takimata sanctus est Lorem ipsum dolor sit amet. Lorem ipsum dolor sit amet, consetetur sadipscing elitr, sed diam nonumy eirmod tempor invidunt ut labore et dolore magna aliquyam erat, sed diam voluptua. At vero eos et accusam et justo duo dolores et ea rebum. Stet clita kasd gubergren, no sea takimata sanctus est Lorem ipsum dolor sit amet. Lorem ipsum dolor sit amet, consetetur sadipscing elitr, sed diam nonumy eirmod tempor invidunt ut labore et dolore magna aliquyam erat, sed diam voluptua. At vero eos et accusam et justo duo dolores et ea rebum. Stet clita kasd gubergren, no sea takimata sanctus est Lorem ipsum dolor sit amet. 

Duis autem vel eum iriure dolor in hendrerit in vulputate velit esse molestie consequat, vel illum dolore eu feugiat nulla facilisis at vero eros et accumsan et iusto odio dignissim qui blandit praesent luptatum zzril delenit augue duis dolore te feugait nulla facilisi. Lorem ipsum dolor sit amet, consectetuer adipiscing elit, sed diam nonummy nibh euismod tincidunt ut laoreet dolore magna aliquam erat volutpat. 

Ut wisi enim ad minim veniam, quis nostrud exerci tation ullamcorper suscipit lobortis nisl ut aliquip ex ea commodo consequat. Duis autem vel eum iriure dolor in hendrerit in vulputate velit esse molestie consequat, vel illum dolore eu feugiat nulla facilisis at vero eros et accumsan et iusto odio dignissim qui blandit praesent luptatum zzril delenit augue duis dolore te feugait nulla facilisi. 

Nam liber tempor cum soluta nobis eleifend option congue nihil imperdiet doming id quod mazim placerat facer possim assum. Lorem ipsum dolor sit amet, consectetuer adipiscing elit, sed diam nonummy nibh euismod tincidunt ut laoreet dolore magna aliquam erat volutpat. Ut wisi enim ad minim veniam, quis nostrud exerci tation ullamcorper suscipit lobortis nisl ut aliquip ex ea commodo consequat. 

Duis autem vel eum iriure dolor in hendrerit in vulputate velit esse molestie consequat, vel illum dolore eu feugiat nulla facilisis. 

At vero eos et accusam et justo duo dolores et ea rebum. Stet clita kasd gubergren, no sea takimata sanctus est Lorem ipsum dolor sit amet. Lorem ipsum dolor sit amet, consetetur sadipscing elitr, sed diam nonumy eirmod tempor invidunt ut labore et dolore magna aliquyam erat, sed diam voluptua. At vero eos et accusam et justo duo dolores et ea rebum. Stet clita kasd gubergren, no sea takimata sanctus est Lorem ipsum dolor sit amet. Lorem ipsum dolor sit amet, consetetur sadipscing elitr, At accusam aliquyam diam diam dolore dolores duo eirmod eos erat, et nonumy sed tempor et et invidunt justo labore Stet clita ea et gubergren, kasd magna no rebum. sanctus sea sed takimata ut vero voluptua. est Lorem ipsum dolor sit amet. Lorem ipsum dolor sit amet, consetetur sadipscing elitr, sed diam nonumy eirmod tempor invidunt ut labore et dolore magna aliquyam erat. 

Consetetur sadipscing elitr, sed diam nonumy eirmod tempor invidunt ut labore et dolore magna aliquyam erat, sed diam voluptua. At vero eos et accusam et justo duo dolores et ea rebum. Stet clita kasd gubergren, no sea takimata sanctus est Lorem ipsum dolor sit amet. Lorem ipsum dolor sit amet, consetetur sadipscing elitr, sed diam nonumy eirmod tempor invidunt ut labore et dolore magna aliquyam erat, sed diam voluptua. At vero eos et accusam et justo duo dolores et ea rebum. Stet clita kasd gubergren, no sea takimata sanctus est Lorem ipsum dolor sit amet. Lorem ipsum dolor sit amet, consetetur sadipscing elitr, sed diam nonumy eirmod tempor invidunt ut labore et dolore magna aliquyam erat, sed diam voluptua. At vero eos et accusam et justo duo dolores et ea rebum. Stet clita kasd gubergren, no sea takimata sanctus. 

Lorem ipsum dolor sit amet, consetetur sadipscing elitr, sed diam nonumy eirmod tempor invidunt ut labore et dolore magna aliquyam erat, sed diam voluptua. At vero eos et accusam et justo duo dolores et ea rebum. Stet clita kasd gubergren, no sea takimata sanctus est Lorem ipsum dolor sit amet. Lorem ipsum dolor sit amet, consetetur sadipscing elitr, sed diam nonumy eirmod tempor invidunt ut labore et dolore magna aliquyam erat, sed diam voluptua. At vero eos et accusam et justo duo dolores et ea rebum. Stet clita kasd gubergren, no sea takimata sanctus est Lorem ipsum dolor sit amet. Lorem ipsum dolor sit amet, consetetur sadipscing elitr, sed diam nonumy eirmod tempor invidunt ut labore et dolore magna aliquyam erat, sed diam voluptua. At vero eos et accusam et justo duo dolores et ea rebum. Stet clita kasd gubergren, no sea takimata sanctus est Lorem ipsum dolor sit amet. 

Duis autem vel eum iriure dolor in hendrerit in vulputate velit esse molestie consequat, vel illum dolore eu feugiat nulla facilisis at vero eros et accumsan et iusto odio dignissim qui blandit praesent luptatum zzril delenit augue duis dolore te feugait nulla facilisi. Lorem ipsum dolor sit amet, consectetuer adipiscing elit, sed diam nonummy nibh euismod tincidunt ut laoreet dolore magna aliquam erat volutpat. 

Ut wisi enim ad minim veniam, quis nostrud exerci tation ullamcorper suscipit lobortis nisl ut aliquip ex ea commodo consequat. Duis autem vel eum iriure dolor in hendrerit in vulputate velit esse molestie consequat, vel illum dolore eu feugiat nulla facilisis at vero eros et accumsan et iusto odio dignissim qui blandit praesent luptatum zzril delenit augue duis dolore te feugait nulla facilisi. 

Nam liber tempor cum soluta nobis eleifend option congue nihil imperdiet doming id quod mazim placerat facer possim assum. Lorem ipsum dolor sit amet, consectetuer adipiscing elit, sed diam nonummy nibh euismod tincidunt ut laoreet dolore magna aliquam erat volutpat. Ut wisi enim ad minim veniam, quis nostrud exerci tation ullamcorper suscipit lobortis nisl ut aliquip ex ea commodo consequat. 

Duis autem vel eum iriure dolor in hendrerit in vulputate velit esse molestie consequat, vel illum dolore eu feugiat nulla facilisis. 

At vero eos et accusam et justo duo dolores et ea rebum. Stet clita kasd gubergren, no sea takimata sanctus est Lorem ipsum dolor sit amet. Lorem ipsum dolor sit amet, consetetur sadipscing elitr, sed diam nonumy eirmod tempor invidunt ut labore et dolore magna aliquyam erat, sed diam voluptua. At vero eos et accusam et justo duo dolores et ea rebum. Stet clita kasd gubergren, no sea takimata sanctus est Lorem ipsum dolor sit amet. Lorem ipsum dolor sit amet, consetetur sadipscing elitr, At accusam aliquyam diam diam dolore dolores duo eirmod eos erat, et nonumy sed tempor et et invidunt justo labore Stet clita ea et gubergren, kasd magna no rebum. sanctus sea sed takimata ut vero voluptua. est Lorem ipsum dolor sit amet. Lorem ipsum dolor sit amet, consetetur sadipscing elitr, sed diam nonumy eirmod tempor invidunt ut labore et dolore magna aliquyam erat. 

Consetetur sadipscing elitr, sed diam nonumy eirmod tempor invidunt ut labore et dolore magna aliquyam erat, sed diam voluptua. At vero eos et accusam et justo duo dolores et ea rebum. Stet clita kasd gubergren, no sea takimata sanctus est Lorem ipsum dolor sit amet. Lorem ipsum dolor sit amet, consetetur sadipscing elitr, sed diam nonumy eirmod tempor invidunt ut labore et dolore magna aliquyam erat, sed diam voluptua. At vero eos et accusam et justo duo dolores et ea rebum. Stet clita kasd gubergren, no sea takimata sanctus est Lorem ipsum dolor sit amet. Lorem ipsum dolor sit amet, consetetur sadipscing elitr, sed diam nonumy eirmod tempor invidunt ut labore et dolore magna aliquyam erat, sed diam voluptua. At vero eos et accusam et justo duo dolores et ea rebum. Stet clita kasd gubergren, no sea takimata sanctus. 

Lorem ipsum dolor sit amet, consetetur sadipscing elitr, sed diam nonumy eirmod tempor invidunt ut labore et dolore magna aliquyam erat, sed diam voluptua. At vero eos et accusam et justo duo dolores et ea rebum. Stet clita kasd gubergren, no sea takimata sanctus est Lorem ipsum dolor sit amet. Lorem ipsum dolor sit amet, consetetur sadipscing elitr, sed diam nonumy eirmod tempor invidunt ut labore et dolore magna aliquyam erat, sed diam voluptua. At vero eos et accusam et justo duo dolores et ea rebum. Stet clita kasd gubergren, no sea takimata sanctus est Lorem ipsum dolor sit amet. Lorem ipsum dolor sit amet, consetetur sadipscing elitr, sed diam nonumy eirmod tempor invidunt ut labore et dolore magna aliquyam erat, sed diam voluptua. At vero eos et accusam et justo duo dolores et ea rebum. Stet clita kasd gubergren, no sea takimata sanctus est Lorem ipsum dolor sit amet. 

Duis autem vel eum iriure dolor in hendrerit in vulputate velit esse molestie consequat, vel illum dolore eu feugiat nulla facilisis at vero eros et accumsan et iusto odio dignissim qui blandit praesent luptatum zzril delenit augue duis dolore te feugait nulla facilisi. Lorem ipsum dolor sit amet, consectetuer adipiscing elit, sed diam nonummy nibh euismod tincidunt ut laoreet dolore magna aliquam erat volutpat. 

Ut wisi enim ad minim veniam, quis nostrud exerci tation ullamcorper suscipit lobortis nisl ut aliquip ex ea commodo consequat. Duis autem vel eum iriure dolor in hendrerit in vulputate velit esse molestie consequat, vel illum dolore eu feugiat nulla facilisis at vero eros et accumsan et iusto odio dignissim qui blandit praesent luptatum zzril delenit augue duis dolore te feugait nulla facilisi. 

Nam liber tempor cum soluta nobis eleifend option congue nihil imperdiet doming id quod mazim placerat facer possim assum. Lorem ipsum dolor sit amet, consectetuer adipiscing elit, sed diam nonummy nibh euismod tincidunt ut laoreet dolore magna aliquam erat volutpat. Ut wisi enim ad minim veniam, quis nostrud exerci tation ullamcorper suscipit lobortis nisl ut aliquip ex ea commodo consequat. 

Duis autem vel eum iriure dolor in hendrerit in vulputate velit esse molestie consequat, vel illum dolore eu feugiat nulla facilisis. 

At vero eos et accusam et justo duo dolores et ea rebum. Stet clita kasd gubergren, no sea takimata sanctus est Lorem ipsum dolor sit amet. Lorem ipsum dolor sit amet, consetetur sadipscing elitr, sed diam nonumy eirmod tempor invidunt ut labore et dolore magna aliquyam erat, sed diam voluptua. At vero eos et accusam et justo duo dolores et ea rebum. Stet clita kasd gubergren, no sea takimata sanctus est Lorem ipsum dolor sit amet. Lorem ipsum dolor sit amet, consetetur sadipscing elitr, At accusam aliquyam diam diam dolore dolores duo eirmod eos erat, et nonumy sed tempor et et invidunt justo labore Stet clita ea et gubergren, kasd magna no rebum. sanctus sea sed takimata ut vero voluptua. est Lorem ipsum dolor sit amet. Lorem ipsum dolor sit amet, consetetur sadipscing elitr, sed diam nonumy eirmod tempor invidunt ut labore et dolore magna aliquyam erat. 

Consetetur sadipscing elitr, sed diam nonumy eirmod tempor invidunt ut labore et dolore magna aliquyam erat, sed diam voluptua. At vero eos et accusam et justo duo dolores et ea rebum. Stet clita kasd gubergren, no sea takimata sanctus est Lorem ipsum dolor sit amet. Lorem ipsum dolor sit amet, consetetur sadipscing elitr, sed diam nonumy eirmod tempor invidunt ut labore et dolore magna aliquyam erat, sed diam voluptua. At vero eos et accusam et justo duo dolores et ea rebum. Stet clita kasd gubergren, no sea takimata sanctus est Lorem ipsum dolor sit amet. Lorem ipsum dolor sit amet, consetetur sadipscing elitr, sed diam nonumy eirmod tempor invidunt ut labore et dolore magna aliquyam erat, sed diam voluptua. At vero eos et accusam et justo duo dolores et ea rebum. Stet clita kasd gubergren, no sea takimata sanctus. 

Lorem ipsum dolor sit amet, consetetur sadipscing elitr, sed diam nonumy eirmod tempor invidunt ut labore et dolore magna aliquyam erat, sed diam voluptua. At vero eos et accusam et justo duo dolores et ea rebum. Stet clita kasd gubergren, no sea takimata sanctus est Lorem ipsum dolor sit amet. Lorem ipsum dolor sit amet, consetetur sadipscing elitr, sed diam nonumy eirmod tempor invidunt ut labore et dolore magna aliquyam erat, sed diam voluptua. At vero eos et accusam et justo duo dolores et ea rebum. Stet clita kasd gubergren, no sea takimata sanctus est Lorem ipsum dolor sit amet. Lorem ipsum dolor sit amet, consetetur sadipscing elitr, sed diam nonumy eirmod tempor invidunt ut labore et dolore magna aliquyam erat, sed diam voluptua. At vero eos et accusam et justo duo dolores et ea rebum. Stet clita kasd gubergren, no sea takimata sanctus est Lorem ipsum dolor sit amet. 

Duis autem vel eum iriure dolor in hendrerit in vulputate velit esse molestie consequat, vel illum dolore eu feugiat nulla facilisis at vero eros et accumsan et iusto odio dignissim qui blandit praesent luptatum zzril delenit augue duis dolore te feugait nulla facilisi. Lorem ipsum dolor sit amet, consectetuer adipiscing elit, sed diam nonummy nibh euismod tincidunt ut laoreet dolore magna aliquam erat volutpat. 

Ut wisi enim ad minim veniam, quis nostrud exerci tation ullamcorper suscipit lobortis nisl ut aliquip ex ea commodo consequat. Duis autem vel eum iriure dolor in hendrerit in vulputate velit esse molestie consequat, vel illum dolore eu feugiat nulla facilisis at vero eros et accumsan et iusto odio dignissim qui blandit praesent luptatum zzril delenit augue duis dolore te feugait nulla facilisi. 

Nam liber tempor cum soluta nobis eleifend option congue nihil imperdiet doming id quod mazim placerat facer possim assum. Lorem ipsum dolor sit amet, consectetuer adipiscing elit, sed diam nonummy nibh euismod tincidunt ut laoreet dolore magna aliquam erat volutpat. Ut wisi enim ad minim veniam, quis nostrud exerci tation ullamcorper suscipit lobortis nisl ut aliquip ex ea commodo consequat. 

Duis autem vel eum iriure dolor in hendrerit in vulputate velit esse molestie consequat, vel illum dolore eu feugiat nulla facilisis. 


%%%%%%%%%%%%%%%%%%%%%%%%%%%%%%%%%%%%%%%%%%%%%%%%%%%%%%%%%%%%%%%%%%%%%%%%%%%%%
%%                                                                         %%
%% /\   /\    Bis hier die Änderungen zur Arbeit vornehmen         /\   /\ %%
%%                                                                         %%
%%%%%%%%%%%%%%%%%%%%%%%%%%%%%%%%%%%%%%%%%%%%%%%%%%%%%%%%%%%%%%%%%%%%%%%%%%%%%

	% Anhang
	\renewcommand{\thetable}{\Alph{section}.\arabic{table}}              % Tabellennummerierung mit Section
	\renewcommand{\thefigure}{\Alph{section}.\arabic{figure}}            % Abbildungsnummerierung mit Section
	\renewcommand{\thelstlisting}{\Alph{section}.\arabic{lstlisting}}    % Listingsnummerierung mit Section



	% Abschluss
	\bibliography{literatur/literatur}
\newpage

	\renewcommand{\glossaryname}{Glossar}
\printglossaries
	\renewcommand{\indexname}{Stichwortverzeichnis}
\printindex
\newpage

	\thispagestyle{empty}
\addcontentsline{toc}{section}{Eidesstattliche Versicherung}
\begin{center}
	\vspace*{2cm}
	\Huge\bf Eidesstattliche Erklärung\\
	\vspace*{3cm}
	\normalsize\rm
	Hiermit versichere ich an Eides statt, dass ich die vorstehende Arbeit selbstständig und ohne fremde Hilfe verfasst habe, dass sie noch an keiner anderen Hochschule, in keinem anderen Studiengang oder sonst als Prüfungsleistung eingereicht wurde und dass ich keine anderen als die angegebenen Quellen und Hilfsmittel benutzt habe. Alle Stellen der Arbeit, die ich wörtlich oder sinngemäß aus Veröffentlichungen oder dergleichen übernommen habe, sind als solche kenntlich gemacht. Ich versichere, dass ich die vorgenannten Angaben nach bestem Wissen und Gewissen gemacht habe und die Angaben der Wahrheit entsprechen. Die Strafbarkeit einer falschen, eidesstattlichen Versicherung ist mir bekannt, selbst bei bloßer Fahrlässigkeit.
	\vfill
	\begin{tabularx}{\textwidth}{l@{\extracolsep\fill}r}
  	\rule{7cm}{0.3mm}&\rule{7.55cm}{0.3mm}\\
	\end{tabularx}
	\begin{tabularx}{\textwidth}{*{2}{>{\arraybackslash}X}}
	  Ort, Datum&Unterschrift\\
	\end{tabularx}
\end{center}


	%\begin{appendix}
	%	\include{chapter/30_anhang}

	%\end{appendix}
\end{document}

%%%%%%%%%%%%%%%%%%%%%%%%%%%%%%%%%%%%%%%%%%%%%%%%%%%%%%%%%%%%%%%%%%%%%%%%%%%%%
%%                                                                         								  %%
%% /\   /\         Ab hier keine Änderungen mehr vornehmen         /\   /\ %%
%%                                                                        								  %%
%%%%%%%%%%%%%%%%%%%%%%%%%%%%%%%%%%%%%%%%%%%%%%%%%%%%%%%%%%%%%%%%%%%%%%%%%%%%%
